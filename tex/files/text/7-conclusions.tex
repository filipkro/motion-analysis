% !TEX root=../../mt-motion-analysis.tex
\chapter{Conclusions and Future work} \label{ch:conclusions}
As discussed in the introduction, the goal with this work was to evaluate whether the \gls{poe} assessments could be automated using these kinds of approaches and to get an idea of what is needed to make that a reality. Based on the results in this thesis, this seems promising, clearly it is possible to assess the \glspl{poe} from the videos. To be able to compare it to other physiotherapists and thereby determine whether the assessments are improved or not, more data is needed. One way to do it could be to have data which surely can be considered correct, preferably labeled by several experts, and then letting physiotherapists perform assessments on the same data. By doing so the performance of the model can be compared to that of the physiotherapists.

One commonly discussed risk when using machine learning methods on data including humans is that unwanted biases from the training data should affect the decisions. Such biases usually includes race of sex, and in this case an example could be that a high ratio of the women have been assessed as having Poor postural orientation. An undesirable behavior would then be for the model to use features such as long hair and certain clothes, typically identifying women, to affect the decision. As no part of the model is trained using the videos as input the risk of such biases are reduced. How to avoid bias in the labeled data is discussed below.


\section{Future work and improvements}
As discussed in Chapter \ref{ch:results}, the performance for the pelvis \gls{poe} assessment seems to be worse than the other \glspl{poe} and this might be due to rotations difficult to see in the 2D joint positions. Hence, it would be interesting to investigate how well this could be done using 3D information. At the initial stages of this work the 3D reconstruction method by Pavllo et al. \cite{Pavllo2019} was briefly evaluated, but a decision was taken to focus on the modeling of the assessments. As about half of the available videos were recorded in a motion capture lab there are 3D information available for these allowing some 3D estimation model to be trained or fine tuned for our movements.

The data used to train and evaluate our method was labeled by one physiotherapist. To make sure bias from this person does not find its way into the data and model assessments from other physiotherapists should be used as well. The current data has been labeled on a per video basis which might introduce bias towards the surrounding repetitions. To split the videos or record single repetition videos and shuffle these before assessment would probably be desirable for the future. When doing this a repetition certainty score could also be gathered which could be used during training. This was evaluated with the certainty data available at the moment as well. However, it did not yield any model improvements which was rather reasonable as this data was gathered for the combined scores and not the repetitions. The ability to give a measure of the model's confidence is important for use in a clinical setting. This can be used to, for instance, ask the patient to repeat the task or, if still uncertain, ask a physiotherapist to assess certain videos and thereby hopefully avoid misclassifications. As of now the output probability of the models seems to be a somewhat good measure for this, but it could potentially be improved by providing the certainty as guidance during training. %As discussed in Chapter \ref{ch:results} this probability is als

Regarding the way the videos are recorded, this method should be rather invariant to factors such as frame rate, orientation, and resolution. However, what is very important for the data collection is to make sure that important body parts are not occluded in the image. As a human unconsciously deduces the positions of occluded objects this might not be considered as important when collecting the data. Hence, this is something to stress when videos are recorded.

With the current approach the lowest hanging fruit for improvements might be the combinations of scores, both combining repetition scores to the final score and the calculation scores. Potentially this could be improved by using some non-linear scoring combination or by learning functions for this as well.



% 3d, mobilapplikation, andra satt att kombinera an average, ngt olijart, trana detta, same om ensembles
